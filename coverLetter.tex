\documentclass[11pt, oneside]{article}   	% use "amsart" instead of "article" for AMSLaTeX format
\synctex=1
\usepackage{geometry}                		% See geometry.pdf to learn the layout options. There are lots.
\geometry{letterpaper}                   		% ... or a4paper or a5paper or ...
%\geometry{landscape}                		% Activate for rotated page geometry
%\usepackage[parfill]{parskip}    		% Activate to begin paragraphs with an empty line rather than an indent
\usepackage{graphicx}				% Use pdf, png, jpg, or eps§ with pdflatex; use eps in DVI mode
								% TeX will automatically convert eps --> pdf in pdflatex		
\usepackage{amssymb}
%\usepackage{subfigure}
\usepackage{subcaption}
\usepackage{amssymb}
\usepackage{amsmath}
\usepackage{hyperref}
%\usepackage{color}


\renewcommand\vec{\mathbf}
\title{Cover letter for PRB submission of ``Higher angular momentum pairing states in Sr$_2$RuO$_4$ in the presence of longer-range interactions''}
\author{}
\date{\today}	
\begin{document}
\maketitle
\noindent
Dear Editor, \\\\
The symmetry of the superconducting order parameter in Sr$_2$RuO$_4$ (SRO) remains a puzzle. 
Recently, time-reversal symmetry breaking $d_{x^2-y^2} \pm i g_{xy(x^2-y^2)}$ pairing was proposed as an order parameter candidate from a phenomenological standpoint to explain a wide variety of key experiments.
The stability of this state, especially of the $g$-wave component, in SRO is unclear. 
It was suggested that it may be stabilized by longer-range interactions based on studies of single-band Hubbard models.  However, a recent study (Ref.~27 in our paper) finds null results when extending this analysis to the SRO models. 
In this paper, we theoretically study the superconducting instabilities in SRO, including the effects of spin-orbit coupling (SOC), in the presence of both local and longer-range interactions within a random phase approximation. 
We show that the inclusion of second nearest neighbor repulsions, together with non-local SOC in the $B_{2g}$ channel or orbital-anisotropy of the non-local interactions, can have a significant impact on the stability of both $d_{x^2-y^2}$- and $g$-wave pairing channels. 
Accidentally / near degenerate $d_{x^2-y^2}+ig$ can be stable for a specific range of parameters.
We further analyze the properties, such as Knight shift and spontaneous edge current, of the realized $d_{x^2-y^2} + ig$ pairing and find that this state is more compatible with existing experimental measurements compared with the $s^{\prime}+id_{xy}$ and mixed helical pairing proposals. 
Our study identifies the key effects that stabilize higher-angular momentum pairing states in SRO, which can also be applied to other multi-band systems, and adds to our understanding of the pairing nature in SRO. 

\noindent
We hope that you will find this paper interesting and substantial, and that it will be accepted for publication as a regular article in Phys. Rev. B. \\\\
\noindent
Sincerely, \\
\noindent
Xin Wang\\
\\\\
\noindent
Possible referees include: \\
\noindent

Peter Joseph Hirschfeld (U of Florida), 
Brian M{\o}ller Andersen (Niels Bohr Institute, U of Copenhagen), 
Srinivas Raghu (Stanford University), 
Thomas Scaffidi (U of Toronto), 
Hae-Young Kee (U of Toronto),
Erez Berg (Weizmann Institute of Science), 
Eun-Ah Kim (Cornell), 
Daniel Agterberg (U Wisconsin-Milwaukee), 
Hong Yao (Tsinghua University)

\end{document}








